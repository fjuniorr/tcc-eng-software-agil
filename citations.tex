\cite{pitac2005}
===

\begin{citacao}
Computational science – the use of advanced computing capabilities to understand and solve complex problems – has become critical to scientific leadership, economic competitiveness, and national security. 

[...]

Computational science provides a unique window through which researchers can investigate problems that are otherwise impractical or impossible to address, ranging from scientific investigations of the biochemical processes of the human brain and the fundamental forces of physics shaping the universe, to analysis of the spread of infectious disease or airborne toxic agents in a terrorist attack, to supporting advanced industrial methods with significant economic benefits, such as rapidly designing more efficient airplane wings computationally rather than through expensive and time-consuming wind tunnel experiments.\cite[pg. iii]{pitac2005}
\end{citacao}


Because the Nation’s research infrastructure has not kept pace with changing technologies, today’s computational science ecosystem is unbalanced, with a software base that is inadequate to keep pace with and support evolving hardware and application needs. By starving research in enabling software and applications, the imbalance forces researchers to build atop inadequate and crumbling foundations rather than on a modern, high-quality software base. The result is greatly diminished productivity for both researchers and computing systems.\cite[pg. 3]{pitac2005}.

The universality of computational science is its intellectual strength. It is also its political weakness. Because all research domains benefit from computational science but none is solely defined by it, the discipline has historically lacked the cohesive, well-organized community of advocates found in other disciplines.\cite[pg. 5, 15]{pitac2005}.

\begin{citacao}
As a basis for responding to the charge from the Office of Science and Technology Policy, the PITAC developed a definition of computational science. This definition recognizes the diverse components, ranging from algorithms, software, architecture, applications, and infrastructure that collectively represent computational science.

Computational science is a rapidly growing multidisciplinary field that uses advanced computing capabilities to understand and solve complex problems. Computational science fuses three distinct elements:

- Algorithms (numerical and non-numerical) and modeling and simulation software developed to solve science (e.g., biological, physical, and social), engineering, and humanities problems
- Computer and information science that develops and optimizes the advanced system hardware, software, networking, and data management components needed to solve computationally demanding problems
- The computing infrastructure that supports both the science and engineering problem solving and the developmental computer and information science
\cite[pg. 3]{pitac2005}.
\end{citacao}

In addition, our preoccupation with peak performance and computing hardware, vital though they are, masks the deeply troubling reality that the most serious technical problems in computational science lie in software, usability, and trained personnel. Heroic efforts are regularly devoted to extending legacy application codes on the latest platforms using primitive software tools and programming models. Meanwhile, the fundamental R&D necessary to create balanced hardware-software systems that are easy to use, facilitate application expression in high-level models, and deliver large fractions of their peak performance on computational science applications is perennially postponed for a more opportune time. More ominously, these difficulties are substantial intellectual hurdles that limit broad education and training. \cite[pg. 16-17]{pitac2005}.

\cite{wilson2013}
===
Scientists spend an increasing amount of time building and using software. However, most scientists are never taught how to do this efficiently. As a result, many are unaware of tools and practices that would allow them to write more reliable and maintainable code with less effort. \cite[pg. 1]{wilson2013}.



