%% abtex2-modelo-artigo.tex, v-1.9.5 laurocesar
%% Copyright 2012-2015 by abnTeX2 group at http://www.abntex.net.br/ 
%%
%% This work may be distributed and/or modified under the
%% conditions of the LaTeX Project Public License, either version 1.3
%% of this license or (at your option) any later version.
%% The latest version of this license is in
%%   http://www.latex-project.org/lppl.txt
%% and version 1.3 or later is part of all distributions of LaTeX
%% version 2005/12/01 or later.
%%
%% This work has the LPPL maintenance status `maintained'.
%% 
%% The Current Maintainer of this work is the abnTeX2 team, led
%% by Lauro César Araujo. Further information are available on 
%% http://www.abntex.net.br/
%%
%% This work consists of the files abntex2-modelo-artigo.tex and
%% abntex2-modelo-references.bib
%%

% ------------------------------------------------------------------------
% ------------------------------------------------------------------------
% abnTeX2: Modelo de Artigo Acadêmico em conformidade com
% ABNT NBR 6022:2003: Informação e documentação - Artigo em publicação 
% periódica científica impressa - Apresentação
% ------------------------------------------------------------------------
% ------------------------------------------------------------------------

\documentclass[
	% -- opções da classe memoir --
	article,			% indica que é um artigo acadêmico
	11pt,				% tamanho da fonte
	oneside,			% para impressão apenas no verso. Oposto a twoside
	a4paper,			% tamanho do papel. 
	% -- opções da classe abntex2 --
	%chapter=TITLE,		% títulos de capítulos convertidos em letras maiúsculas
	%section=TITLE,		% títulos de seções convertidos em letras maiúsculas
	%subsection=TITLE,	% títulos de subseções convertidos em letras maiúsculas
	%subsubsection=TITLE % títulos de subsubseções convertidos em letras maiúsculas
	% -- opções do pacote babel --
	english,			% idioma adicional para hifenização
	brazil,				% o último idioma é o principal do documento
	sumario=tradicional
	]{abntex2}


% ---
% PACOTES
% ---

% ---
% Pacotes fundamentais 
% ---
\usepackage{lmodern}			% Usa a fonte Latin Modern
\usepackage[T1]{fontenc}		% Selecao de codigos de fonte.
\usepackage[utf8]{inputenc}		% Codificacao do documento (conversão automática dos acentos)
\usepackage{indentfirst}		% Indenta o primeiro parágrafo de cada seção.
\usepackage{nomencl} 			% Lista de simbolos
\usepackage{color}				% Controle das cores
\usepackage{graphicx}			% Inclusão de gráficos
\usepackage{microtype} 			% para melhorias de justificação
% ---
		
% ---
% Pacotes adicionais, usados apenas no âmbito do Modelo Canônico do abnteX2
% ---
\usepackage{lipsum}				% para geração de dummy text
% ---
		
% ---
% Pacotes de citações
% ---
\usepackage[brazilian,hyperpageref]{backref}	 % Paginas com as citações na bibl
\usepackage[alf]{abntex2cite}	% Citações padrão ABNT
% ---

% ---
% Configurações do pacote backref
% Usado sem a opção hyperpageref de backref
\renewcommand{\backrefpagesname}{Citado na(s) página(s):~}
% Texto padrão antes do número das páginas
\renewcommand{\backref}{}
% Define os textos da citação
\renewcommand*{\backrefalt}[4]{
	\ifcase #1 %
		Nenhuma citação no texto.%
	\or
		Citado na página #2.%
	\else
		Citado #1 vezes nas páginas #2.%
	\fi}%
% ---

% ---
% Informações de dados para CAPA e FOLHA DE ROSTO
% ---
\titulo{Práticas Ágeis para o Desenvolvimento de Software Científico}
\autor{Francisco Alves} 
\local{Brasil}
\data{2015}
% ---

% ---
% Configurações de aparência do PDF final

% alterando o aspecto da cor azul
\definecolor{blue}{RGB}{41,5,195}

% informações do PDF
\makeatletter
\hypersetup{
     	%pagebackref=true,
		pdftitle={\@title}, 
		pdfauthor={\@author},
    	pdfsubject={Modelo de artigo científico com abnTeX2},
	    pdfcreator={LaTeX with abnTeX2},
		pdfkeywords={abnt}{latex}{abntex}{abntex2}{atigo científico}, 
		colorlinks=true,       		% false: boxed links; true: colored links
    	linkcolor=blue,          	% color of internal links
    	citecolor=blue,        		% color of links to bibliography
    	filecolor=magenta,      		% color of file links
		urlcolor=blue,
		bookmarksdepth=4
}
\makeatother
% --- 

% ---
% compila o indice
% ---
\makeindex
% ---

% ---
% Altera as margens padrões
% ---
\setlrmarginsandblock{3cm}{3cm}{*}
\setulmarginsandblock{3cm}{3cm}{*}
\checkandfixthelayout
% ---

% --- 
% Espaçamentos entre linhas e parágrafos 
% --- 

% O tamanho do parágrafo é dado por:
\setlength{\parindent}{1.3cm}

% Controle do espaçamento entre um parágrafo e outro:
\setlength{\parskip}{0.2cm}  % tente também \onelineskip

% Espaçamento simples
\SingleSpacing

% ----
% Início do documento
% ----
\begin{document}

% Seleciona o idioma do documento (conforme pacotes do babel)
%\selectlanguage{english}
\selectlanguage{brazil}

% Retira espaço extra obsoleto entre as frases.
\frenchspacing 

% ----------------------------------------------------------
% ELEMENTOS PRÉ-TEXTUAIS
% ----------------------------------------------------------

%---
%
% Se desejar escrever o artigo em duas colunas, descomente a linha abaixo
% e a linha com o texto ``FIM DE ARTIGO EM DUAS COLUNAS''.
% \twocolumn[    		% INICIO DE ARTIGO EM DUAS COLUNAS
%
%---
% página de titulo
\maketitle

% resumo em português
\begin{resumoumacoluna}
 Conforme a ABNT NBR 6022:2003, o resumo é elemento obrigatório, constituído de
 uma sequência de frases concisas e objetivas e não de uma simples enumeração
 de tópicos, não ultrapassando 250 palavras, seguido, logo abaixo, das palavras
 representativas do conteúdo do trabalho, isto é, palavras-chave e/ou
 descritores, conforme a NBR 6028. (\ldots) As palavras-chave devem figurar logo
 abaixo do resumo, antecedidas da expressão Palavras-chave:, separadas entre si por
 ponto e finalizadas também por ponto.
 
 \vspace{\onelineskip}
 
 \noindent
 \textbf{Palavras-chave}: latex. abntex. editoração de texto.
\end{resumoumacoluna}

% ]  				% FIM DE ARTIGO EM DUAS COLUNAS
% ---

% ----------------------------------------------------------
% ELEMENTOS TEXTUAIS
% ----------------------------------------------------------
\textual

% ----------------------------------------------------------
% Introdução
% ----------------------------------------------------------
\section*{Introdução}
\addcontentsline{toc}{section}{Introdução}
A evolução da infra-estrutura computacional disponível para pesquisadores nos mais diversos campos do conhecimento tem gerado o desenvolvimento de duas novas disciplinas intrinsicamente relacionadas que visam utilizar estes recursos para a resolução de problemas, quais sejam, a ciência computacional e a ciência dos dados.

De acordo com \citeonline{pitac2005} a ciência computacional pode ser entendida como um campo de estudos interdisciplinar que visa a utilização de recursos computacionais avançados para o entendimento e resolução de problemas complexos. Ao contrário da ciência da computação, que visa o estudo da computação a partir de uma perspectiva científica, o foco da ciência computacional é a resolução de problemas de outras disciplinas científicas, o foco é a aplicação.

Por sua vez, uma das definições mais aceitas de ciência dos dados provém de \citeonline{cleveland2014} que utilizou o termo para se referir a uma proposta de expansão das áreas de interesse da estatística com foco na análise de dados. Dentre as áreas de expansão sugeridas encontram-se computação com dados e avaliação de ferramentas.

% Melhorar a definição de data science

Estes dois campos estão tão relacionados e vem ganhando tanta força, que tem sido chamados, respectivamente, de terceiro e quartos pilares da ciência, ao lado da teoria e da experimentação. Apesar posições contrárias a inclusão desses campos como pilares da ciência, como aquela de \citeonline{vardi2010}, é possível encontrar no mínimo dois pontos de unanimidade: 1) a ciência computacional e a ciência dos dados estão proporcionando avanço científico, econômico e social a ponto de serem consideradas uma revolução; 2) cada vez mais pesquisadores irão precisar de habilidades computacionais para aproveitar os novos recursos a disposição para resolução de seus problemas.

% Necessário citações para justificar afirmativas 1 e 2

Este trabalho irá se debruçar sobre uma das componentes dessa revolução que se faz presente tanto na ciência computacional quanto na ciência de dados. O desenvolvimento de software. Mais especificamente, o tema deste artigo é a utilização de práticas ágeis de engenharia de software por pesquisadores para o desenvolvimento de softwares científicos.

A justificativa para este trabalho deriva da importância destes novos campos para o avanço científico e da constatação de que pesquisadores gastam uma parcela cada vez maior do seu tempo desenvolvendo software sem o treinamento adequado. \cite{wilson2014}. Isso gera pesquisadores autodidatas que não utilizam práticas de engenharia de software que já se tornaram \emph{mainstream} na indústria. O conselho de \citeonline{gentzkow2014} é extremamente válido nesse contexto:  

\begin{citacao}
If you are trying to solve a problem, and there are multi-billion dollar firms whose entire business model depends on solving the same problem, and there are whole courses at your university devoted to how to solve that problem, you might want to figure out what the experts do and see if you can’t learn something from it. \cite[pg. 5]{gentzkow2014}.
\end{citacao}

Além disso, o processo de desenvolvimento de software científico aparenta ter características similares a aquelas endereçadas pelo manifesto ágil, como responsividade a mudança e colaboração \cite{sletholt2012}, tornando essas práticas especialmente interessantes de serem estudadas para fins de utilização por pesquisadores. Deste modo, o problema de pesquisa desta trabalho pode ser formulado como: Quais práticas ágeis de engenharia de software são adequadas as necessidades dos pesquisadores que precisam desenvolver software no âmbito de sua pesquisa?

O objetivo geral deste trabalho é identificar quais práticas ágeis de engenharia de software podem ser utilizadas por pesquisadores para o desenvolvimento de softwares científicos. Para tanto, podem ser listados como objetivos específicos deste trabalho:

- Documentar as práticas de engenharia de software caracterizadas pela agilidade

- Avaliar a similaridade entre o contexto de trabalho de engenheiros de software e pesquisadores

- Identificar quais práticas podem ser utilizadas com poucas adaptações

- Identificar práticas que não podem ser utilizadas tendo em vista a diferença entre os contextos

Pode-se encontrar na literatura especializada incontáveis e absolutamente diversas classificações que se aplicam a metodologia. No presente trabalho, quatro critérios de classificação serão adotados: aquele em relação a natureza, em relação aos objetivos gerais da pesquisa, em relação a abordagem empregada – se qualitativa ou quantitativa, e aquele em relação aos métodos empregados.

Em relação a natureza trata-se de uma pesquisa aplicada que visa gerar soluções para problemas específicos relacionados a aplicação de práticas de engenharia de software para a pesquisa científica. A abordagem aplicada será qualitativa caracterizada pelo aprofundamento nas questões subjetivas do fenômeno em detrimento da produção de medidas quantitativas. Quanto aos objetivos será uma pesquisa exploratória tendo em vista que seu objetivo é buscar familiaridade com problemas pouco conhecidos. Quanto aos procedimentos técnicos será uma pesquisa bibliográfica especialmente por meio de artigos científicos que descrevam as práticas de engenharia de software e o contexto do desenvolvimento de softwares científicos.

% ----------------------------------------------------------
% Seção de explicações
% ----------------------------------------------------------
\section*{Revisão da Literatura}

\subsection*{Desenvolvimento de Software Científico}
\label{subsec:scientific_software_development}
O objetivo desta seção é caracterizar o processo de desenvolvimento de software científico, com foco especial nas particularidades que existem em função do produto final (ie: software científico) e dos indivíduos envolvidos (ie: pesquisadores) neste processo.

O software científico pode ser entendido como aquele desenvolvido por pesquisadores com o objetivo de resolução de um problema em sua área de conhecimento de interesse. Apesar de simples, essa definição tem encontrado respaldo na literatura. \citeonline{sletholt2012} define software científico como sendo aquele desenvolvido por cientistas e para cientistas, seja para a resolução de sistemas de equações matemáticas ou para geração de simulações. \citeonline{heaton2015} segue a mesma linha ao afirmar que software científico é aquele desenvolvido por cientistas e engenheiros para substituir ou expandir a experimentação física.

De maneira geral, um processo de desenvolvimento de software pode ser entendido como um conjunto de atividades relacionadas que levam a produção de um software. No caso da produção de software científico, a regra parece ser a ausência de um processo formal, com a realização de tarefas ad-hoc que geram o resultado desejado. Além da ausência de um processo formal, algumas das peculiaridades já documentadas na literatura dizem respeito ao levantamento de requisitos; a ausência de testes; a forma de relacionamento com o cliente; e ao treinamento recebido pelos responsáveis por desenvolver softwares científicos.

Em relação ao levantamento de requisitos, duas características principais se destacam. Em primeiro lugar, os requisitos de um software científico aparentam ser especialmente mais indefinidos no início de um projeto em comparação com softwares tradicionais. É usual que o problema de pesquisa de um cientista evolua ao longo de sua pesquisa, e, consequentemente, os requisitos acompanham essa transformação. Apesar de requisitos que emergem ao longo do tempo também existirem em contextos de softwares tradicionais, é razoável supor que ao menos o problema é mais bem definido e irá mudar menos em contextos tradicionais do que no científico. Além disso, uma segunda característica que explica as particularidades do levantamento de requisitos de softwares cientíticos deriva do fato de que os desenvolvedores, em grande medida cientistas, possuem vasto domínio de negócio. Isso implica que os requisitos podem ser levantados de forma extremamente genérica, tendo em vista que o indíviduo responsável pela implementação possui conhecimento suficiente para conceber o que deveria ser implementado em primeiro lugar. Nas palavras de um pesquisador que estava acostumado a desenvolver software e depois a especificar:

\begin{citacao}
So all I told one person [the developer] was: I want you to find a way of doing a ... fast graph matching problem, in an interface that is easy to use and shows you everything you need to know on the interface, and that’s all I said. And he went away for a year and came back and here is the system. \cite{segal2008b}.
\end{citacao}

O processo de teste de software científico também apresenta diferenças marcantes em relação ao processo de teste de softwares tradicionais. Em primeiro lugar, uma das diferenças que será explorada adiante também ocorre no contexto de testes, qual seja, pesquisadores não são, via de regra, treinados em como testar software. Além disso, a característica exploratória de diversos softwares científicos faz com que seja difícil definir, \emph{a priori}, qual o comportamento esperado de um programa. Por fim, o teste de algoritmos númericos de estimação apresentam complexidades inerentes a forma como computadores armazenam números flutuantes. As passagens abaixo capturam a essência de testes em um ambiente de software científico:

\begin{citacao}
Testing is of the cursory nature which would enrage a software engineer (‘does the software do what I expect it to do with inputs of the type I would expect to use? If I don’t have any real expectations of the output, does the software behave in a way I really would not expect given inputs of the type I would expect to use?’). \cite{segal2008b}.
\end{citacao}

% \begin{citacao}
% It is argued by many philosophers and historians of science, see for example, Chalmers, 1982, that scientists assume that their instruments work unless confronted by absolutely incontrovertible evidence. Perhaps this assumption also holds for their software: the innate quality of the software is not questioned unless it becomes clear that the software is not supporting the science. \cite{segal2008a}.
% \end{citacao}

Em relação ao relacionamento com os clientes, o \emph{modus operandi} tradicional de desenvolvimento de software científico é que um pesquisador que possui familiaridade com o desenvolvimento de software, provavelmente em virtude de contato durante a elaboração da sua tese de doutorado, é responsável por desenvolver os softwares de um grupo de pesquisa ou de um laboratório. Ou seja, o desenvolvedor não só possui conhecimento da área de negócio, ele trabalha diariamente com os demais clientes.

% Falar sobre processo decisório sem figura de autoridade formal \cite{Agile methods in biomedical software development: a multi-site experience report}

A última grande diferença é relativa ao treinamento (ou falta dele) que os responsáveis pelo desenvolvimento de software científico usualmente possuem. Enquanto no caso do desenvolvimento de software tradicionais os responsáveis usualmente possuem treinamento específico em ciência da computação e engenharia de software, a realidade de pesquisadores é que eles nunca são treinados em como implementar software. \citeonline{wilson2014} observa que estudos recentes mostram que cientistas tipicamente gastam 30\% ou mais do seu tempo desenvolvendo software mas 90\% ou mais são autodidatas, e, portanto, nunca foram expostos a práticas básicas de engenharia de software como controle de versão, revisão de código, teste unitário e automação de tarefas. 

Em relação ao treinamento de cientistas, merece destaque o Software Carpentry \url{https://software-carpentry.org/}, iniciativa que visa treinar cientistas em habilidades computacionais básicas que permitem um ganho de eficiência por parte de pesquisadores envolvidos em computação científica.

\subsection*{Práticas Ágeis de Desenvolvimento de Software}
As características apresentadas na última seção objetivaram mostrar as particularidades do processo de desenvolvimento de software científico que devem ser reconhecidas por qualquer prática de engenharia de software que pretenda ser efetiva no referido contexto. O objetivo desta seção é apresentar um conjunto de práticas que aparentam se moldar especialmente bem as necessidades de pesquisadores. A essas deu-se o nome de práticas ágeis.

O movimento de agilidade teve início formal com a publicação do manifesto ágil em 2001. Na ocasião, um grupo de 17 profissionais se reuniram com o objetivo de encontrar similaridades entre os processos de desenvolvimento de software advogados por cada um dos participantes. Todos as metodologias se apresentavam como uma alternativa aos processos cascata de desenvolvimento bastantes difundidos na época. A integra do manifesto é

\begin{citacao}
We are uncovering better ways of developing software by doing it and helping others do it. Through this work we have come to value:

Individuals and interactions over processes and tools
Working software over comprehensive documentation
Customer collaboration over contract negotiation
Responding to change over following a plan

That is, while there is value in the items on the right, we value the items on the left more.
\end{citacao}

Além do manifesto, também foram publicados doze princípios que dão sustenção aos valores expostos pelos agilistas. No entanto, o objetivo deste trabalho é debruçar-se sobre as práticas ágeis, entendidas como as atividades concretas realizadas no dia a dia para desenvolvimento e gestão de projetos de software de maneira consistente com o manifesto ágil. A justificativa para essa abordagem deriva do entendimento que pesquisadores não irão se interessar pelo movimento ágil como um todo a menos que sejam capazes de observar melhorias concretas no seu dia a dia advindas das práticas aqui expostas.

As práticas ágeis listadas abaixo foram extraídas do Guia de Práticas Ágeis \url{http://guide.agilealliance.org/} publicado pela \emph{Agile Alliance}. Quando necessário, foram pesquisados outras referências bibliográficas para melhor caracterização da mesma, bem como a forma pela qual as especificidades do software científico são bem atendidas. As práticas foram agrupadas de acordo com as principais particularidades existentes no processo de desenvolvimento de software científico conforme identificado na seção \ref{subsec:scientific_software_development}, além de um tópico adicional destinado a agrupar práticas que podem ser consideradas como boas práticas em engenharia de software em geral não necessariamente associadas ao movimento ágil.

\subsubsection*{Requisitos}

\paragraph*{\emph{Backlog}}
O \emph{backlog} é uma lista com itens que representa todo o trabalho que deve ou pode ser realizado em um dado projeto. Ele apresenta uma característica evolutiva por natureza, ao representar tudo o que o software pode vir a ser com base nas definições advindas dos stakeholders (clientes, usuários, desenvolvedores, etc). A característica evolutiva advém do fato de que a qualquer momento do tempo novos itens podem ser inseridos ou removidos do \emph{backlog} com base em novas prioridades e requisitos. Não existe um formato físico pré-definido para o backlog e nem um nível de granularidade para os itens a ele pertencentes.

\paragraph*{\emph{Design} Simples}
O termo \emph{Design} Simples faz referência ao fato de que a estratégia de definição do \emph{design} do software deve ser baseado no reconhecimento de que essa atividade é recorrente, que sua qualidade deve ser baseada na sua simplicidade e que decisões de design devem ser realizadas no último momento responsivo. Outro termo utilizado para essa prática é design emergente, enfocando o fato de que um design de qualidade irá emergir ao prestarmos atenção as qualidades locais do código.

\subsubsection*{Testes}

\paragraph*{Testes Unitários}
Testes unitários são pequenos trechos de código que visam exercitar o código sob teste para garantir que o mesmo se comporta da forma esperada. Usualmente os testes unitários são escritos na mesma linguagem de programação do código sob teste, testam uma funcionalidade restrita do código sob teste, explicando o adjetivo unitário, e geralmente rodam de maneira extremamente rápida.

\paragraph*{Desenvolvimento Orientado a Testes - TDD}
O Desenvolvimento orientado a testes representa um estilo de programação onde o teste unitário é escrito antes do código sob teste, gerando uma situação de falha inicial na primeira execução do teste. Posteriormente a código sob teste é escrito até que o teste unitário passe. Por fim, o código sob teste é refatorado como firma de manter sua simplicidade.

\subsubsection*{Relacionamento com Cliente}

\paragraph*{Time}
O time em um contexto ágil representa um grupo pequeno de pessoas, até nove pessoas no SCRUM, que trabalham juntas para a concretização de um projeto. A idéia de time tenta conjurar a idéia de que nenhum indíviduo é pessoalmente responsável pelo sucesso ou fracasso de um projeto, mas sim o time como um todo. Além disso, todas as habilidades necessárias para finalização do projeto devem estar presentes no time, sejam elas técnicas ou ligadas ao negócio.

\paragraph*{Reunião Diária}
Diariamente o time se reune para compartilhar informações em relação ao progresso do trabalho até então. Essa reunião é breve, no máximo 15 minutos, e é estruturada em torno de três perguntas principais, quais sejam: O que você completou desde a última reunião?; O que você planeja completar até a próxima reunião?; e Quais impedimentos você encontrou?

\subsubsection*{Práticas Fundamentais}

\paragraph*{Controle de Versão}

\paragraph*{Desenvolvimento Iterativo e Incremental}

\paragraph*{Build Automatizado}


\subsubsection*{Treinamento}

\paragraph*{Programação em Pares}

% ---
% Finaliza a parte no bookmark do PDF, para que se inicie o bookmark na raiz
% ---
\bookmarksetup{startatroot}% 
% ---

% ---
% Conclusão
% ---
\section*{Considerações finais}
\addcontentsline{toc}{section}{Considerações finais}
Este trabalho visou identificar práticas ágeis de engenharia de software que podem ser utilizadas por pesquisadores para o desenvolvimento de softwares científicos.

A caracterização do processo de desenvolvimento de software científico 

% ----------------------------------------------------------
% ELEMENTOS PÓS-TEXTUAIS
% ----------------------------------------------------------
\postextual

% ---
% Título e resumo em língua estrangeira
% ---

% \twocolumn[    		% INICIO DE ARTIGO EM DUAS COLUNAS

% titulo em inglês
\titulo{Agile Practices for Scientific Software Development}
\emptythanks
\maketitle

% resumo em português
\renewcommand{\resumoname}{Abstract}
\begin{resumoumacoluna}
 \begin{otherlanguage*}{english}
   According to ABNT NBR 6022:2003, an abstract in foreign language is a back
   matter mandatory element.

   \vspace{\onelineskip}
 
   \noindent
   \textbf{Keywords}: latex. abntex.
 \end{otherlanguage*}  
\end{resumoumacoluna}

% ]  				% FIM DE ARTIGO EM DUAS COLUNAS
% ---

% ----------------------------------------------------------
% Referências bibliográficas
% ----------------------------------------------------------
\bibliography{references}

% ----------------------------------------------------------
% Glossário
% ----------------------------------------------------------
%
% Há diversas soluções prontas para glossário em LaTeX. 
% Consulte o manual do abnTeX2 para obter sugestões.
%
%\glossary

% ----------------------------------------------------------
% Apêndices
% ----------------------------------------------------------

% ---
% Inicia os apêndices
% ---
%\begin{apendicesenv}

% ----------------------------------------------------------
%\chapter{Nullam elementum urna vel imperdiet}
% ----------------------------------------------------------

%\end{apendicesenv}
% ---

% ----------------------------------------------------------
% Anexos
% ----------------------------------------------------------
\cftinserthook{toc}{AAA}
% ---
% Inicia os anexos
% ---
%\anexos
%\begin{anexosenv}

% ---
%\chapter{Cras non urna sed feugiat}
% ---

%\end{anexosenv}

\end{document}
